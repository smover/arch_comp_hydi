\documentclass[11pt,twoside,a4paper]{article}

\usepackage{hyperref}

\begin{document}
\title{Piecewise Constant Navigation Benchmark}
\author{Alessandro Cimatti, Alberto Griggio, Sergio Mover, Stefano Tonetta}
\maketitle

\section{Introduction}

The benchmarks is inspired by the classic navigation benchmark
from~\cite{HSCC04}, but the goals are quite different.

The proposed benchmark differs from the one in~\cite{HSCC04} for the 
kind of dynamics (piecewise constants and not affine) used to
describe the motion of the vehicles in each cell and also the
coordinate systems used is different.

The benchmark is significant for the following reasons.
\begin{enumerate}
\item It contains non-deterministic piece-wise constants dynamics, that change in each discrete location.
%
The class of PWC hybrid automata allow to model non-deterministic
 dynamics that have several applications in the
modeling of real-time systems and as a high level abstraction of
hybrid systems.

The proposed model differs from the other PWC ARCH
benchmark~\cite{ttethernetarch}, which only contains clock variables.

%% \item Non-deterministic initial states: the model is more complex
%% since it has a set of initial states and not a single one. This is
%% often the case when requirements are not clear.

%% \item It requires to solve different kind of problems:
%% \begin{itemize}
%%   \item verify safety and reachability properties.
%%   \item bounded reachability.
%%   \item synthesis of parameters.
%% \end{itemize}
\item Scalable: the complexity of the model can be increased by
  increasing the number of cells. A scalable model allow us to compare
  the performance of different verification engines.
\end{enumerate}

We provide the benchmarks in SpaceEx and hydi format in a public
github repository: \url{https://github.com/smover/arch_comp_hydi}


\section{Benchmark description}

The model describes an object that moves on the 2d plane.
%
The plane is divided into cells by a grid with the same number of rows
and columns, as shown in Figure~\ref{fig:example}.
%
The continuous variables $x$ and $y$ represent the position of the
object on the $x$ and $y$ axes.

\begin{figure}
\begin{verbatim}

               |     |     |          |    |
           SE  | SE  | SE  |          | SEE| NWW
               |     |     |          |    |
     ----------+-----+-----+--      --+----+-----------------
           NEE | NEE | NEE |  other   | NEE| NE
               |     |     |  columns |    |
    -----------+-----+-----|--      --+----+-----------------
           NEE | NEE | NEE |          | NEE| NE
               |     |     |          |    |
    -----------+-----+-----+--      --+----------------------

                         ... other rows...

               |     |     |          |    |
    -----------+-----+-----+--      --+----------------------
           NEE | NEE | NEE |          | NEE| NE
               |     |     |          |
\end{verbatim}

\caption{nxn version of the navigation benchmark}
\end{figure}

TODO: add an image of the grid, with the directions

In each cell the object moves following a specified dynamic that
constraints the velocity in the $x$ and $y$ directions.

The dynamic can be one among the following:
\begin{itemize}
\item N (North): $\dot{x} = 0    , \dot{y} = 1.4$
\item NE (Nort-East): $\dot{x} = 1.4  , \dot{y} = 1.4$
\item E (East): $\dot{x} = 1.4  , \dot{y} = 0$
\item SE (South-East): $\dot{x} = 1.4  , \dot{y} = -1.4$
\item S (South): $\dot{x} = 0    , \dot{y} = -1.4$
\item SW (South-West): $\dot{x} = -1.4 , \dot{y} = -1.4$
\item W (West): $\dot{x} = -1.4 , \dot{y} = 0$
\item NW (North-West): $\dot{x} = -1.4 , \dot{y} = 1.4$
\item NE-E (North-East East), the object moves non-deterministally in
  the range NE - E: $\dot{x} = 1.4, 0 <= \dot{y}, \dot{y} <= 1.4$
\item SE-S (South-East South), the object moves non-deterministally in
  the range SE - S: $0 <= \dot{x}, \dot{x} <= 1.4, \dot{y} = -1.4$
\item NW-W (North-West West), the object moves non-deterministally in
  the range NW - W: $\dot{x} = -1.4, 0 <= \dot{y}, \dot{y} <= 1.4$
\item FREE, the object moves non-deterministically in
  all the possible direction, with a maximum rate of $1$:
  $-1 <= \dot{x}, \dot{x}  <= 1, -1 <= \dot{y}, \dot{y}  <= 1$
\end{itemize}

The objects starts in $x=0$ and $y=0$ coordinates.
The directions used in the cells are such that the object will
eventually reach the 4 top right cells. Then, the object is forced to
move inside these cells.

The benchmark is scaled by increasing the size of the grid.


\section{Verification problem}

\begin{itemize}
\item Safety property: the safety property states that after a
  specific amount of time the object must stay inside the 4 top-right
  cells.
\end{itemize}





\end{document}
